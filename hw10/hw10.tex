\documentclass{article}
\usepackage{amsfonts, amsthm, amsmath, amssymb, mathtools, ulem, mathrsfs, physics, esint, siunitx, tikz-cd}
\usepackage{pdfpages, fullpage, color, microtype, cancel, textcomp, markdown, hyperref, graphicx}
\usepackage{enumitem}
\graphicspath{{./images/}}
\usepackage[english]{babel}
\usepackage[autostyle, english=american]{csquotes}
\MakeOuterQuote{"}
\usepackage{xparse}
\usepackage{tikz}
\usepackage{algpseudocode}

% fonts
\def\mbb#1{\mathbb{#1}}
\def\mfk#1{\mathfrak{#1}}
\def\mbf#1{\mathbf{#1}}
\def\tbf#1{\textbf{#1}}

% common bold letters
\def\bP{\mbb{P}}
\def\bC{\mbb{C}}
\def\bH{\mbb{H}}
\def\bI{\mbb{I}}
\def\bR{\mbb{R}}
\def\bQ{\mbb{Q}}
\def\bZ{\mbb{Z}}
\def\bN{\mbb{N}}

% brackets
\newcommand{\br}[1]{\left(#1\right)}
\newcommand{\sbr}[1]{\left[#1\right]}
\newcommand{\brc}[1]{\left\{#1\right\}}
\newcommand{\lbr}[1]{\left\langle#1\right\rangle}

% matrices
\newcommand{\m}[2][b]{\begin{#1matrix}#2\end{#1matrix}}
\newcommand{\arr}[3][\sbr]{#1{\begin{array}{#2}#3\end{array}}}

% misc
\NewDocumentCommand{\app}{O{x} O{\infty}}{\xrightarrow{#1\to#2}}
\renewcommand{\ss}{\subset}
\newcommand{\vn}{\varnothing}
\newcommand{\inv}{^{-1}}
\newcommand{\imp}{\implies}
\newcommand{\impleft}{\reflectbox{$\implies$}}
\renewcommand{\bar}{\overline}
\renewcommand{\d}{\partial}
\newcommand{\pf}{\tbf{Proof. }}

% greek
\newcommand{\e}{\epsilon}
\newcommand{\p}{\varphi}
\renewcommand{\t}{\theta}
\renewcommand{\a}{\alpha}
\renewcommand{\l}{\lambda}

% title
\title{Scientific Computing HW10}
\author{Ryan Chen}
%\date{\today}
\setlength{\parindent}{0pt}


\begin{document}
	
\maketitle
	


\begin{enumerate}
	
	
	
	\item Exercise 3.5 of [NW] as a lemma: If $B$ is nonsingular then
	\[\norm{Bx} \ge \frac{\norm{x}}{\norm{B\inv}}\]
	\pf First $\norm{x} = \norm{B\inv Bx} \le \norm{B\inv}\norm{Bx}$, then divide by $\norm{B\inv}$. \qed
	
	The direction is given by
	\[p = -B\inv\grad f(x)\]
	We can rewrite this as
	\[\grad f(x) = -Bp\]
	Given SPD $B$ we can define $B^{1/2}$ and $B^{-1/2}$, moreover $\norm{B^{1/2}}=\norm{B}^{1/2}$ and $\norm{B^{-1/2}}=\norm{B}^{-1/2}$. Then compute
	\begin{align*}
		\cos\t &= -\frac{\grad f(x)^Tp}{\norm{\grad f(x)}\norm{p}} \\
		&= \frac{p^TBp}{\norm{Bp}\norm{p}} \\
		&\ge \frac{p^TBp}{\norm{B}\norm{p}^2} & \norm{Bp} \le \norm{B}\norm{p} \\
		&= \frac{\norm{B^{1/2}p}^2}{\norm{B}\norm{p}^2} & p^TBp = p^TB^{1/2}B^{1/2}p = \norm{B^{1/2}p}^2 \\
		&= \frac{\norm{p}^2}{\norm{B^{-1/2}}^2\norm{B}\norm{p}^2} & \text{by lemma} \\
		&= \frac{1}{\norm{B\inv}\norm{B}} \\
		&= \frac{1}{\kappa(B)} \\
		&> 0
	\end{align*}
	This establishes $\cos\t$ as bounded away from zero.
	
	
	\pagebreak
	
	
	
	\item
	
	\begin{enumerate}
		
		
		
		\item Compute
		\begin{align*}
			y_k^Ts_k &= (\grad f_{k+1}^T - \grad f_k^T)\a_kp_k \\
			&= \a_k(\grad f_{k+1}^Tp_k - \grad f_k^Tp_k) \\
			&\ge \a_k(c_2\grad f_k^Tp_k - \grad f_k^Tp_k) & \text{Wolfe condition 2} \\
			&= \a_k(c_2-1)\grad f_k^Tp_k \\
			&> 0 & \text{$\grad f_k^Tp_k$ by Problem 1, and $c_2<1$}
		\end{align*}
		
		
		
		\item The inner product and norm induced by $B_k$ are
		\[(u,v)_{B_k} := v^TB_ku,
		\quad \norm{u}_{B_k} := \sqrt{(u,u)_{B_k}}\]
		For all $z\ne0$,
		\begin{align*}
			z^TB_{k+1}z &= \norm{z}_{B_k}^2 - \frac{(z,s_k)_{B_k}^2}{\norm{s_k}_{B_k}^2} + \frac{(z^Ty_k)^2}{y_k^Ts_k} \\
			&> \norm{z}_{B_k}^2 - \frac{(z,s_k)_{B_k}^2}{\norm{s_k}_{B_k}^2} & y_k^Ts_k > 0 \\
			&\ge \norm{z}_{B_k}^2 - \frac{\norm{z}_{B_k}^2\norm{s_k}_{B_k}^2}{\norm{s_k}_{B_k}^2} & \text{Cauchy--Schwarz} \\
			&= 0
		\end{align*}
		Thus $B_{k+1}$ is SPD.
		
		
		
	\end{enumerate}



	\pagebreak
	
	
	
	\item
	
	
	
	\pagebreak
	
	
	
	\item
	
	\begin{enumerate}
		
		
		
		
		\item Compute
		\[\d_xf = 200(y-x^2)(-2x) + 2(1-x)(-1) = -400x(y-x^2) + 2(x-1)\]
		\[\d_yf = 200(y-x^2)\]
		Critical points are found by solving $\d_xf=\d_yf=0$. From $\d_yf=0$ we get $y=x^2$. Plugging this into $\d_xf=0$ gives $x=1$. Thus the only critical point is $(1,1)$.
		
		The Hessian is
		\[H(x,y) = \m{\d_x^2f & \d_x\d_yf \\ \d_x\d_yf & \d_y^2f} = \m{-400(y-3x^2)+2 & -400x \\ -400x & 200}\]
		Evaluate it at the critical point.
		\[H(1,1) = \m{802 & -400 \\ -400 & 200}\]
		Its eigenvalues $\l_1,\l_2$ are given by
		\[\l_1+\l_2 = \tr H(1,1) = 1002\]
		\[\l_1\l_2 = \det H(1,1) = 400\]
		From $\l_1+\l_2>0$ and $\l_1\l_2>0$ we have $\l_1,\l_2>0$ hence $H(1,1)$ is SPD. We conclude that the only critical point $(1,1)$ is a local minimizer.
		 
		
		
		\item
		
		
		
		
	\end{enumerate}
	
	
	
\end{enumerate}
	
	
	
\end{document}