\documentclass{article}
\usepackage{amsfonts, amsthm, amsmath, amssymb, mathtools, ulem, mathrsfs, physics, esint, siunitx, tikz-cd}
\usepackage{pdfpages, fullpage, color, microtype, cancel, textcomp, markdown, hyperref, graphicx}
\usepackage{enumitem}
\graphicspath{{./images/}}
\usepackage[english]{babel}
\usepackage[autostyle, english=american]{csquotes}
\MakeOuterQuote{"}
\usepackage{xparse}
\usepackage{tikz}
\usepackage{algpseudocode}

% fonts
\def\mbb#1{\mathbb{#1}}
\def\mfk#1{\mathfrak{#1}}
\def\mbf#1{\mathbf{#1}}
\def\tbf#1{\textbf{#1}}

% common bold letters
\def\bP{\mbb{P}}
\def\bC{\mbb{C}}
\def\bH{\mbb{H}}
\def\bI{\mbb{I}}
\def\bR{\mbb{R}}
\def\bQ{\mbb{Q}}
\def\bZ{\mbb{Z}}
\def\bN{\mbb{N}}

% brackets
\newcommand{\br}[1]{\left(#1\right)}
\newcommand{\sbr}[1]{\left[#1\right]}
\newcommand{\brc}[1]{\left\{#1\right\}}
\newcommand{\lbr}[1]{\left\langle#1\right\rangle}

% matrices
\newcommand{\m}[2][b]{\begin{#1matrix}#2\end{#1matrix}}
\newcommand{\arr}[3][\sbr]{#1{\begin{array}{#2}#3\end{array}}}

% misc
\NewDocumentCommand{\app}{O{x} O{\infty}}{\xrightarrow{#1\to#2}}
\renewcommand{\ss}{\subset}
\newcommand{\vn}{\varnothing}
\newcommand{\inv}{^{-1}}
\newcommand{\imp}{\implies}
\newcommand{\impleft}{\reflectbox{$\implies$}}
\renewcommand{\bar}{\overline}
\renewcommand{\d}{\partial}
\newcommand{\pf}{\tbf{Proof. }}

% greek
\newcommand{\e}{\epsilon}
\newcommand{\p}{\varphi}
\renewcommand{\t}{\theta}
\renewcommand{\l}{\lambda}
\newcommand{\D}{\Delta}

% title
\title{Scientific Computing HW 11}
\author{Ryan Chen}
%\date{\today}
\setlength{\parindent}{0pt}


\begin{document}
	
\maketitle
	


\begin{enumerate}
	
	
	
	\item 
	
	\begin{enumerate}
		
		
		
		\item If $v=0$ then $\ip{u}{v}=0$ and $\ip{u}{u}\ip{v}{v}=0$, so the Cauchy--Schwarz inequality holds. Now assume $v\ne0$. The function
		\[p(t) := \ip{u+tv}{u+tv},~ t\in\bR\]
		is nonnegative due to positive--definiteness of the inner product. Expand $p(t)$ as
		\[p(t) = \ip{v}{v}t^2 + 2\ip{u}{v}t + \ip{u,u}\]
		which is a quadratic in $t$ due to positive--definiteness $(\ip{v}{v}>0)$. Its discriminant is
		\[\Delta = 4\ip{u}{v}^2 - 4\ip{u}{u}\ip{v}{v} = 4(\ip{u}{v}^2-\ip{u}{u}\ip{v}{v})\]
		Since $p$ is nonnegative, it does not change sign, hence $\Delta\le0$, from which the Cauchy--Schwarz inequality follows.
		
		
		
		\item Since $B$ is SPD, we can define the matrices $B^{1/2}$ and $B^{-1/2}$. Then
		\begin{align*}
			(g^TBg)(g^TB\inv g) &= (g^TB^{1/2}B^{1/2}g)(g^TB^{-1/2}B^{-1/2}g)\\
			&= \ip{B^{1/2}g}{B^{1/2}g}\ip{B^{-1/2}g}{B^{-1/2}g} \\
			&\ge \ip{B^{1/2}g}{B^{-1/2}g}^2 & \text{by Cauchy--Schwarz} \\
			&= (g^TB^{1/2}B^{-1/2}g)^2 \\
			&= (g^Tg)^2
		\end{align*}
		
		
		
	\end{enumerate}
	
	
	
	
	
	
	\pagebreak
	
	
	
	
	
	
	\item For convenience write $\l=\l^{(l)}$. To prove that the iterations are equivalent, we must show that
	\[-\frac{\phi(\l)}{\phi'(\l)} = \frac{\norm{p_l}^2}{\norm{z_l}^2}\frac{\norm{p_l}-\D}{\D}\]
	
	First we establish some equalities. Observe that
	\begin{align*}
		\frac{q_j^Tg}{\l_j+\l} &= [(B+\l I)\inv q_j]^Tg \\
		&= q_j^T(B+\l I)\inv g \\
		&= -q_j^Tp_l
	\end{align*}
	hence
	\begin{equation}\label{eq:2.1}
		\frac{(q_j^Tg)^2}{(\l_j+\l)^2} = (q_j^Tp_l)^2
	\end{equation}
	Also observe that
	\begin{align*}
		\frac{q_j^Tg}{(\l_j+\l)^{1/2}} &= [(B+\l I)^{-1/2} q_j]^Tg \\
		&= q_j^T (B+\l I)^{-1/2} g \\
		&= q_j^T L\inv p_l \\
		&= q_j^T z_l
	\end{align*}
	This, along with \eqref{eq:2.1}, implies
	\begin{equation}\label{eq:2.2}
		\frac{(q_j^Tg)^2}{(\l_j+\l)^3} = \frac{(q_j^Tp_l)^2}{\l_j+\l} = (q_j^Tp_l)^2
	\end{equation}	
	Since $B$ is SPD, we may take the $q_j$'s to form an orthonormal basis of $\bR^n$, so that
	\begin{equation}\label{eq:2.3}
		\norm{x}^2 = \sum_j (q_j^Tx)^2,~ x\in\bR^n
	\end{equation}
	
	Rewrite $\phi(\l)$.
	\begin{align*}
		\phi(\l) &= \D\inv - \sbr{\sum_j \frac{(q_j^Tg)^2}{(\l_j+\l)^2}}^{-1/2} \\
		&= \D\inv - \sbr{\sum_j (q_j^Tp_l)}^{-1/2} & \text{by \eqref{eq:2.1}} \\
		&= \D\inv - \norm{p_l}\inv & \text{by \eqref{eq:2.3}}
	\end{align*}
	Obtain and rewrite $\phi'(\l)$.
	\begin{align*}
		\phi'(\l) &= -\sbr{\sum_j \frac{(q_j^Tg)^2}{(\l_j+\l)^2}}^{-3/2}\sum_j \frac{(q_j^Tg)^2}{(\l_j+\l)^3} \\
		&= -\sbr{\sum_j (q_j^Tp_l)^2}^{-3/2} \sum_j (q_j^Tz_l)^2 & \text{by \eqref{eq:2.1} and \eqref{eq:2.2}} \\
		&= -\norm{p_l}^{-3}\norm{z_l}^2 & \text{by \eqref{eq:2.3}}
	\end{align*}
	Finally, we have
	\begin{align*}
		-\frac{\phi(\l)}{\phi'(\l)} &= \frac{\D\inv - \norm{p_l}\inv}{\norm{p_l}^{-3}\norm{z_l}^2} \\
		&= \frac{\D\inv - \norm{p_l}\inv}{\norm{p_l}^{-3}\norm{z_l}^2} \cdot \frac{\D\norm{p_l}}{\D\norm{p_l}} \\
		&= \frac{\norm{p_l}-\D}{\norm{p_l}^{-2}\norm{z_l}^2\D} \\
		&= \frac{\norm{p_l}^2}{\norm{z_l}^2}\frac{\norm{p_l}-\D}{\D}
	\end{align*}

	
	
	
	
	\pagebreak
	
	
	
	
	
	
	\item
	
	
	
	
	
	
	\pagebreak
	
	
	
	
	
	
	\item
	
	
	
\end{enumerate}
	
	
\end{document}