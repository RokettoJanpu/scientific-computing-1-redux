\documentclass{article}
\usepackage{amsfonts, amsthm, amsmath, amssymb, mathtools, ulem, mathrsfs, physics, esint, siunitx, tikz-cd}
\usepackage{pdfpages, fullpage, color, microtype, cancel, textcomp, markdown, hyperref, graphicx}
\usepackage{enumitem}
\graphicspath{{./images/}}
\usepackage[english]{babel}
\usepackage[autostyle, english=american]{csquotes}
\MakeOuterQuote{"}
\usepackage{xparse}
\usepackage{tikz}

% fonts
\def\mbb#1{\mathbb{#1}}
\def\mfk#1{\mathfrak{#1}}
\def\mbf#1{\mathbf{#1}}
\def\tbf#1{\textbf{#1}}

% common bold letters
\def\bP{\mbb{P}}
\def\bC{\mbb{C}}
\def\bH{\mbb{H}}
\def\bI{\mbb{I}}
\def\bR{\mbb{R}}
\def\bQ{\mbb{Q}}
\def\bZ{\mbb{Z}}
\def\bN{\mbb{N}}

% brackets
\newcommand{\br}[1]{\left(#1\right)}
\newcommand{\sbr}[1]{\left[#1\right]}
\newcommand{\brc}[1]{\left\{#1\right\}}
\newcommand{\lbr}[1]{\left\langle#1\right\rangle}

% matrices
\newcommand{\m}[2][b]{\begin{#1matrix}#2\end{#1matrix}}
\newcommand{\arr}[3][\sbr]{#1{\begin{array}{#2}#3\end{array}}}
\DeclareMathOperator{\Span}{span}

% greek
\newcommand{\e}{\epsilon}
\newcommand{\p}{\varphi}
\renewcommand{\t}{\theta}
\renewcommand{\l}{\lambda}
\renewcommand{\u}{\mu}

% misc
\NewDocumentCommand{\app}{O{x} O{\infty}}{\xrightarrow{#1\to#2}}
\newcommand{\sse}{\subseteq}
\renewcommand{\ss}{\subset}
\newcommand{\vn}{\varnothing}
\newcommand{\inv}{^{-1}}
\newcommand{\imp}{\implies}
\newcommand{\impleft}{\reflectbox{$\implies$}}
\renewcommand{\ip}[2]{\lbr{#1,#2}}
\renewcommand{\bar}{\overline}
\DeclareMathOperator{\cis}{cis}
\DeclareMathOperator{\Arg}{Arg}
\renewcommand{\d}{\partial}
\newcommand{\pf}{\tbf{Proof. }}

% title
\title{Scientific Computing HW 4}
\author{Ryan Chen}
%\date{\today}
\setlength{\parindent}{0pt}


\begin{document}
	
\maketitle



\begin{enumerate}
	
	
	
	\item e
	
	
	
	\pagebreak
	
	
	
	\item Before answering each part, we cite the following
	\[\grad(x^TAx) = (A+A^T)x,
	\quad \grad(x^Tx) = 2x\]
	We compute
	\[\grad Q(x) = \frac{(x^Tx)(A+A^T)x - (x^TAx)2x}{(x^Tx)^2}\]
	Thus
	\[\grad Q(x) = 0
	\iff (x^Tx)(A+A^T)x = 2(x^TAx)x
	\iff (A+A^T)x = \frac{2x^TAx}{x^Tx}x\]
	
	\begin{enumerate}
		
		
		
		
		\item Under the condition $A$ is symmetric, the above computation gives
		\[\grad Q(x) = 0
		\iff Ax = \frac{x^TAx}{x^Tx}x
		\iff \br{\frac{x^TAx}{x^Tx},x} \text{ is an eigenpair of $A$}\]
		
		
		
		\item If $A$ is asymmetric, the above calculation says $\grad Q(x)=0$ iff $\br{\frac{2x^TAx}{x^Tx},x}$ is an eigenpair of $A+A^T$.
		
		
		
		
	\end{enumerate}



	\pagebreak
	
	
	
	\item
	
	\begin{enumerate}
		
		
		
		\item Lemma: If $A$ is nonsingular and $(\l,v)$ is an eigenpair of $A$ then $(\l\inv,v)$ is an eigenpair of $A\inv$.
		
		Proof of lemma: From $A$ being nonsingular, $\l\ne0$. Then
		\[Av = \l v
		\imp v = \l A\inv v
		\imp A\inv v = \l\inv v\]
		
		Since $\u$ is not an eigenvalue of $A$, we know $A-\u I$ is nonsingular. Then
		\[(A-\u I)v = Av - \u Iv
		= \l v - \u v
		= (\l-\u)v\]
		hence $((\l-\u),v)$ is an eigenpair of $A-\u I$. By the lemma, $((\l-\u)\inv,v)$ is an eigenpair of $(A-\u I)\inv$.
		
		
		
		\item
		
		
		
		\item
		
		
		
	\end{enumerate}
	
	
	
\end{enumerate}
	
	
\end{document}