\documentclass{article}
\usepackage{amsfonts, amsthm, amsmath, amssymb, mathtools, ulem, mathrsfs, physics, esint, siunitx, tikz-cd}
\usepackage{pdfpages, fullpage, color, microtype, cancel, textcomp, markdown, hyperref, graphicx}
\usepackage{enumitem}
\graphicspath{{./images/}}
\usepackage[english]{babel}
\usepackage[autostyle, english=american]{csquotes}
\MakeOuterQuote{"}
\usepackage{xparse}
\usepackage{tikz}
\usepackage{algpseudocode}

% fonts
\def\mbb#1{\mathbb{#1}}
\def\mfk#1{\mathfrak{#1}}
\def\mbf#1{\mathbf{#1}}
\def\tbf#1{\textbf{#1}}

% common bold letters
\def\bP{\mbb{P}}
\def\bC{\mbb{C}}
\def\bH{\mbb{H}}
\def\bI{\mbb{I}}
\def\bR{\mbb{R}}
\def\bQ{\mbb{Q}}
\def\bZ{\mbb{Z}}
\def\bN{\mbb{N}}

% brackets
\newcommand{\br}[1]{\left(#1\right)}
\newcommand{\sbr}[1]{\left[#1\right]}
\newcommand{\brc}[1]{\left\{#1\right\}}
\newcommand{\lbr}[1]{\left\langle#1\right\rangle}

% matrices
\newcommand{\m}[2][b]{\begin{#1matrix}#2\end{#1matrix}}
\newcommand{\arr}[3][\sbr]{#1{\begin{array}{#2}#3\end{array}}}

% misc
\NewDocumentCommand{\app}{O{x} O{\infty}}{\xrightarrow{#1\to#2}}
\renewcommand{\ss}{\subset}
\newcommand{\vn}{\varnothing}
\newcommand{\inv}{^{-1}}
\newcommand{\imp}{\implies}
\newcommand{\impleft}{\reflectbox{$\implies$}}
\renewcommand{\bar}{\overline}
\renewcommand{\d}{\partial}
\newcommand{\pf}{\tbf{Proof. }}

% greek
\newcommand{\e}{\epsilon}
\newcommand{\p}{\varphi}
\renewcommand{\t}{\theta}

% title
\title{Scientific Computing HW 9}
\author{Ryan Chen}
%\date{\today}
\setlength{\parindent}{0pt}


\begin{document}
	
\maketitle
	


\begin{enumerate}
	
	
	
	\item 
	
	\begin{enumerate}
		
		
		
		\item Label the three factors in the claimed factorization as $L,S,U$. Compute
		\[LS = \m{L_{11} & & \\ & L_{22} & \\ A_{31}U_{11}\inv & A_{32}U_{22}\inv & S_{33}}
		\imp LSU = \m{A_{11} & & A_{13} \\ & A_{22} & A_{23} \\ A_{31} & A_{32} & B}\]
		where
		\[B := A_{31}U_{11}\inv L_{11}\inv A_{13} + A_{32}U_{22}\inv L_{22}\inv A_{23} + S_{33}\]
		Imposing $A=LSU$ gives a formula for the Schur complement $S_{33}$.
		\begin{align*}
			A = LSU
			&\iff B = A_{33} \\
			&\iff A_{31}U_{11}\inv L_{11}\inv A_{13} + A_{32} U_{22}\inv L_{22}\inv A_{23} + S_{33} = A_{33} \\
			&\iff S_{33} = A_{33} - A_{31}U_{11}\inv L_{11}\inv A_{13} - A_{32} U_{22}\inv L_{22}\inv A_{23}
		\end{align*}

		
	
	
		\item Given $S_{33}=L_{33}U_{33}$, the LU decomposition of $A$ is
		\[A = \m{L_{11} & & \\ & L_{22} & \\ A_{31}U_{11}\inv & A_{32}U_{22}\inv & L_{33}}
			  \m{U_{11} & & L_{11}\inv A_{13} \\ & U_{22} & L_{22}\inv A_{23} \\ & & U_{33}}\]
		
		
		
	\end{enumerate}



	\pagebreak
	
	
	
	\item Code: \url{https://github.com/RokettoJanpu/scientific-computing-1-redux/blob/main/hw9/hw9.ipynb}
	
	The norm of difference between methods is 1200. This seems too large; perhaps there is an error in how I set up the problem.
	
	
	
	\pagebreak
	
	
	
	\item The algorithm does not compute the Cholesky decomposition of $A_{11}$ or $A_{22}$ until the grid size is small, so we only consider the costs of computing the Schur complement $S_{33}$ and its Cholesky decomposition on each level.
	
	We will focus on level 0. The block sizes of the tessellation of $A$ are
	\begin{align*}
		A_{11},A_{22}&: \frac N2\times\frac N2 \\
		A_{13},A_{23}&: \frac N2\times N^{1/2} \\
		A_{31},A_{32}&: N^{1/2}\times\frac N2 \\
		A_{33}&: N^{1/2}\times N^{1/2}
	\end{align*}
	We compute $S_{33}$ via
	\[S_{33} = A_{33} - A_{31}U_{11}\inv L_{11}\inv A_{13} - A_{32} U_{22}\inv L_{22}\inv A_{23}\]
	According to the block sizes, a product involving the inverse of a triangular matrix, e.g. $L_{11}\inv A_{13}$, may be considered as solving a sparse nonsingular triangular $\frac N2\times\frac N2$ system with $N^{1/2}$ RHS vectors, and there are 4 products to compute. Being sparse gives roughly $\frac N2$ nonzero entries, and being nonsingular triangular forces at least $\frac N2$ of the nonzero entries to lie on the diagonal, so the system is roughly diagonal. Thus the cost is $\frac N2N^{1/2}4=2N^{3/2}$.
	
	The cost of computing $AB$, where $A$ is $m\times k$ sparse and $B$ is $k\times n$ possibly nonsparse, is roughly $2nnz(A)ncols(B)$ with $nnz(A)$ the number of nonzero entries of $A$ and $ncols(B)$ the number of columns of $B$. There are two products involving a sparse matrix and a possibly nonsparse matrix, e.g. $A_{31}(U_{11}\inv L_{11}\inv A_{13})$. Considering the block sizes and $A_{31}$ sparse with roughly $\frac N2$ nonzero entries, estimate the cost as $2\frac N2N^{1/2}2=2N^{3/2}$.
	
	There are two sums of $N^{1/2}\times N^{1/2}$ possibly nonsparse matrices to compute, contributing $2N$ to the cost. Thus the total cost of computing $S_{33}$ is $4N^{3/2}$. Then the cost of computing the Cholesky decomposition of $S_{33}$, it being $N^{1/2}\times N^{1/2}$, is $\frac13N^{3/2}$. In conclusion, the cost of level 0 is $aN^{3/2}$ where $a:=\frac{13}{3}$.
	
	Now consider the cost on level $k$. The meshgrid is partitioned with $2^k$ "dividers" of length $\br{\frac{N}{2^k}}^{1/2}$ each. Applying the reasoning behind the cost of level 0, the cost of level $k$ is $a2^k\br{\frac{N}{2^k}}^{3/2} = aN^{3/2}(2^{-1/2})^k$. Thus the cost of the entire algorithm is estimated as
	\[W \le aN^{3/2}\sum_{k=0}^\infty (2^{-1/2})^k = \frac{13}{3}\frac{1}{1-2^{-1/2}}N^{3/2}\]
	
	
	
\end{enumerate}
	
	
\end{document}